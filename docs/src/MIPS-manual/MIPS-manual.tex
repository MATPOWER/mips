%! program = pdflatex

\documentclass[12pt]{article}

%%% PACKAGES
\usepackage{ifpdf}
\usepackage{booktabs}   % for much better looking tables
\usepackage{array}      % for better arrays (eg matrices) in maths
\usepackage{paralist}   % very flexible & customisable lists (eg. enumerate/itemize, etc.)
%\usepackage{verbatim}   % adds environment for commenting out blocks of text & for better verbatim
\usepackage{subfigure}  % make it possible to include more than one captioned figure/table in a single float
% These packages are all incorporated in the memoir class to one degree or another...
\ifpdf
\usepackage[pdftex]{graphicx}
\else
\usepackage[dvips]{graphicx}
\fi
\usepackage{epstopdf}
\usepackage[cmex10]{amsmath}
\interdisplaylinepenalty=2500

%%% PAGE DIMENSIONS
\usepackage{geometry} % to change the page dimensions
\geometry{letterpaper}
%\geometry{margins=2in} % for example, change the margins to 2 inches all round
%\geometry{landscape} % set up the page for landscape
% read geometry.pdf for detailed page layout information

%% HEADERS & FOOTERS
\usepackage{fancyhdr} % This should be set AFTER setting up the page geometry
\pagestyle{fancy} % options: empty , plain , fancy
\renewcommand{\headrulewidth}{0pt} % customise the layout...
%\lhead{}\chead{}\rhead{\hl{--- DRAFT ---}}
\lhead{}\chead{}\rhead{}
\lfoot{}\cfoot{\thepage}\rfoot{}

% *** MY ADDITIONAL PACKAGES ***
\usepackage{amsfonts}
\usepackage{amstext}
%\usepackage{ctable}        % messes up table captions among other things
\usepackage{booktabs}       % defines \toprule, \midrule, \bottomrule
\usepackage{threeparttable} % needed for table notes
\usepackage{longtable}      % for multi-page tables
\usepackage[font=normalsize]{caption}   % to keep caption for multi-page
                            % table normal size after using \footnotesize
                            % to reduce rest of table
%\usepackage{multirow}
%\usepackage{mathenv}
\usepackage{textcomp}      % improves \textregistered, provides \textquotesingle
\usepackage[usenames]{color}
\usepackage{soul}
\usepackage{fancyvrb}
\usepackage{relsize}
\usepackage[noadjust]{cite} % prevent adding a space
%\usepackage{url}
\usepackage{xr-hyper}
\usepackage[usenames,dvipsnames,svgnames,table]{xcolor} % additional named colors
\usepackage[colorlinks=true,urlcolor=blue,hyperfootnotes=false,backref=section,citecolor=LimeGreen]{hyperref}
%\usepackage[colorlinks=true,urlcolor=blue,hyperfootnotes=false]{hyperref}
% These are (supposedly) the defaults
% \hypersetup{
%     bookmarks=true,         % show bookmarks bar?
%     unicode=false,          % non-Latin characters in Acrobat's bookmarks
%     pdftoolbar=true,        % show Acrobat's toolbar?
%     pdfmenubar=true,        % show Acrobat's menu?
%     pdffitwindow=true,      % page fit to window when opened
%     pdftitle={My title},    % title
%     pdfauthor={Author},     % author
%     pdfsubject={Subject},   % subject of the document
%     pdfcreator={Creator},   % creator of the document
%     pdfproducer={Producer}, % producer of the document
%     pdfkeywords={keywords}, % list of keywords
%     pdfnewwindow=true,      % links in new window
%     colorlinks=false,       % false: boxed links; true: colored links
%     linkcolor=red,          % color of internal links
%     citecolor=green,        % color of links to bibliography
%     filecolor=magenta,      % color of file links
%     urlcolor=cyan           % color of external links
% }
\usepackage{upquote}        % single quotes in verbatim environments
\usepackage[title,titletoc]{appendix}

%%% SECTION TITLE APPEARANCE
% \usepackage{sectsty}
% \allsectionsfont{\sffamily\mdseries\upshape} % (See the fntguide.pdf for font help)
% (This matches ConTeXt defaults)

%%% ToC APPEARANCE
\usepackage[nottoc,notlof,notlot]{tocbibind} % Put the bibliography in the ToC
% \usepackage[titles]{tocloft} % Alter the style of the Table of Contents
% \renewcommand{\cftsecfont}{\rmfamily\mdseries\upshape}
% \renewcommand{\cftsecpagefont}{\rmfamily\mdseries\upshape} % No bold!

% Namelist environment generates a list with an item width of your choice.
% Syntax:  \begin{namelist}{widthstring} .  From Buerger's book.
\newcommand{\namelistlabel}[1]{\mbox{#1}\hfil}
\newenvironment{namelist}[1]{%
\begin{list}{}
  {
    \let\makelabel\namelistlabel
    \settowidth{\labelwidth}{#1}
    \setlength{\leftmargin}{1.1\labelwidth}
  }
 }{%
\end{list}}

% define my verbatim environment, requires packages fancyvrb & relsize
\renewcommand{\FancyVerbFormatLine}[1]{\makebox[2mm][l]{}#1}
\DefineVerbatimEnvironment%
  {Code}{Verbatim}
  {fontsize=\relsize{-1.5},
  samepage=true,
  frame=single}

% define my verbatim environment, requires packages fancyvrb & relsize
\renewcommand{\FancyVerbFormatLine}[1]{\makebox[2mm][l]{}#1}
\DefineVerbatimEnvironment%
  {Notice}{Verbatim}
  {fontsize=\relsize{-1.5},
  samepage=true,
  xleftmargin=15mm,
  framesep=5mm,
  frame=single}

%% makes verbatim text 'small' (using verbatim package)
%\makeatletter
%\g@addto@macro\@verbatim\small
%\makeatother

%\hyphenation{matpower_manual}

\newcommand{\mpver}[0]{7.1}
%\newcommand{\matlab}[0]{{Matlab}}
%\newcommand{\matlab}[0]{{\sc Matlab}\textsuperscript{\tiny \textregistered}}
\newcommand{\matlab}[0]{{\sc Matlab}}
\newcommand{\matpower}[0]{{\sc Matpower}}
\newcommand{\matpowerurl}[0]{https://matpower.org}
\newcommand{\matpowerlink}[0]{\href{\matpowerurl}{\matpower{}}}
\newcommand{\matpowergithuburl}[0]{https://github.com/MATPOWER/matpower}
\newcommand{\mptest}[0]{{MP-Test}}
\newcommand{\mptesturl}[0]{https://github.com/MATPOWER/mptest}
\newcommand{\mptestlink}[0]{\href{\mptesturl}{\mptest{}}}
\newcommand{\mptestver}[0]{7.1}
\newcommand{\mips}[0]{{MIPS}}
\newcommand{\mipsurl}[0]{https://github.com/MATPOWER/mips}
\newcommand{\mipslink}[0]{\href{\mipsurl}{\mips{}}}
\newcommand{\mipsname}[0]{{{\bf M}{\sc atpower} \textbf{I}nterior \textbf{P}oint \textbf{S}olver}}
\newcommand{\mipsver}[0]{1.4}
\newcommand{\mpom}[0]{\mbox{MP-Opt-Model}}
\newcommand{\mpomurl}[0]{https://github.com/MATPOWER/mp-opt-model}
\newcommand{\mpomlink}[0]{\href{\mpomurl}{\mpom{}}}
\newcommand{\mpomname}[0]{\mpom{}}
% \newcommand{\mpomname}[0]{{{\bf M}{\sc at}{\bf P}{\sc ower} \textbf{Opt}imization \textbf{Model}}}
\newcommand{\mpomver}[0]{3.0}
\newcommand{\most}[0]{{MOST}}
\newcommand{\mostname}[0]{{{\bf M}{\sc atpower} \textbf{O}ptimal \textbf{S}cheduling \textbf{T}ool}}
\newcommand{\mosturl}[0]{https://github.com/MATPOWER/most}
\newcommand{\mostver}[0]{1.1}
\newcommand{\md}[0]{{\most{} Data struct}}
\newcommand{\powerweb}[0]{{\sc PowerWeb}}
\newcommand{\pserc}[0]{{\sc PSerc}}
\newcommand{\PSERC}[0]{{Power Systems Engineering Research Center (\pserc{})}}
\newcommand{\certs}[0]{{\sc Certs}}
\newcommand{\CERTS}[0]{{Consortium for Electric Reliability Technology Solutions (\certs{})}}
\newcommand{\ipopt}[0]{{\sc Ipopt}}
\newcommand{\knitro}[0]{{Artelys Knitro}}
\newcommand{\clp}[0]{{CLP}}
\newcommand{\cplex}[0]{{CPLEX}}
\newcommand{\glpk}[0]{{GLPK}}
\newcommand{\gurobi}[0]{{Gurobi}}
\newcommand{\mosek}[0]{{MOSEK}}
\newcommand{\ot}[0]{{Optimization Toolbox}}
\newcommand{\pardiso}[0]{{PARDISO}}
\newcommand{\sdppf}[0]{\texttt{SDP\_PF}}
\newcommand{\sdppfver}[0]{1.0.2}
\newcommand{\sdpopf}[0]{SDPOPF}
\newcommand{\code}[1]{{\relsize{-0.5}{\tt{{#1}}}}}  % requires package relsize
% Note: to get straight single quotes in \code you have to use one of the
%       following: \char13 \char'15 \char"0D \textquotesingle
\newcommand{\codeq}[1]{\code{\textquotesingle{}#1\textquotesingle}}  % requires package textcomp
\newcommand{\mppath}[1]{\textsf{\textsl{{\relsize{-1.0}\textless{}\mbox{MATPOWER}\textgreater{}}}}\code{{#1}}}  % requires package relsize
%\newcommand{\mostpath}[1]{\code{\$MOST{#1}}}
\newcommand{\mipspath}[1]{\textsf{\textsl{{\relsize{-1.0}\textless{}\mbox{MIPS}\textgreater{}}}}\code{{#1}}}  % requires package relsize
\newcommand{\mostpath}[1]{\mppath{}\code{/most{#1}}}
\newcommand{\baseMVA}[0]{\code{baseMVA}}
\newcommand{\bus}[0]{\code{bus}}
\newcommand{\branch}[0]{\code{branch}}
\newcommand{\gen}[0]{\code{gen}}
\newcommand{\gencost}[0]{\code{gencost}}
\newcommand{\areas}[0]{\code{areas}}
\newcommand{\mpc}[0]{\code{mpc}}
\newcommand{\results}[0]{\code{results}}
\newcommand{\mumurl}[0]{https://matpower.org/docs/MATPOWER-manual-\mpver.pdf}
\newcommand{\mum}[0]{\href{\mumurl}{\matpower{} User's Manual}}
\newcommand{\mipsmanurl}[0]{https://matpower.org/docs/MIPS-manual-\mipsver.pdf}
\newcommand{\mostmanurl}[0]{https://matpower.org/docs/MOST-manual-\mostver.pdf}
\newcommand{\currentmumurl}[0]{https://matpower.org/docs/MATPOWER-manual.pdf}
\newcommand{\currentmipsmanurl}[0]{https://matpower.org/docs/MIPS-manual.pdf}
\newcommand{\currentmostmanurl}[0]{https://matpower.org/docs/MOST-manual.pdf}
\newcommand{\mipsman}[0]{\href{\mipsmanurl}{\mips{} User's Manual}}
\newcommand{\mostman}[0]{\href{\mostmanurl}{\most{} User's Manual}}
\newcommand{\dg}[0]{\sp\dagger}                         % hermitian conjugate
\newcommand{\trans}[1]{{#1}^{\ensuremath{\mathsf{T}}}}  % transpose
%\newcommand{\trans}[1]{#1^{\ensuremath{\mathsf{T}}}}    % transpose
\newcommand{\cc}[1]{{#1}^{\ast}}                        % complex conjugate
\newcommand{\hc}[1]{{#1}^{\dg}}                         % hermitian conjugate
\newcommand{\conj}[1]{{#1}^{+}}                         % hermitian conjugate
\newcommand{\diag}[1]{\left[{#1}\right]}                % diagonal
\newcommand{\R}{\mathbb{R}}          % requires \usepackage{amsfonts|bbold}
\newcommand{\der}[2]{\frac{\partial{#1}}{\partial{#2}}} % partial derivative
\newcommand{\doi}[1]{doi:~\href{https://doi.org/#1}{#1}}

\def\sectionautorefname{Chapter}
\def\subsectionautorefname{Section}
\def\subsubsectionautorefname{Section}
\newcommand{\secref}[1]{\autoref{#1} \nameref{#1}}

\numberwithin{equation}{section}
\numberwithin{table}{section}
\renewcommand{\thetable}{\thesection\mbox{-}\arabic{table}}
\numberwithin{figure}{section}
\renewcommand{\thefigure}{\thesection\mbox{-}\arabic{figure}}

% \externaldocument[MOSTMAN-]{MOST-manual}[\mostmanurl]


%\title{\hl{--- DRAFT  ---}\\\hl{\em do not distribute}\\~\\\mipsname{}\\\mips{} \mipsver{}\\
\title{\mipsname{}\\\mips{} \mipsver{}\\
User's Manual}
\author{Ray~D.~Zimmerman \and Hongye~Wang}
%\date{October 8, 2020} % comment this line to display the current date
%\date{December 14, 2011\thanks{Second revision. First revision was December 13, 2011}} % comment this line to display the current date

%%% BEGIN DOCUMENT
\begin{document}
\maketitle
\thispagestyle{empty}
\vfill
\begin{center}
{\scriptsize
\copyright~2009--2020~\PSERC{}\\
All Rights Reserved}
\end{center}

\clearpage
%\setcounter{page}{2}
\tableofcontents
\clearpage
%\listoffigures
\listoftables

%%------------------------------------------
\clearpage
\section{Introduction}

\subsection{Background}

\mipsname{} (\mipslink{}) is a package of \matlab{} language M-files\footnote{Also compatible with GNU Octave~\cite{octave}.} for solving non-linear programming problems (NLPs) using a primal dual interior point method. The \mips{} project page can be found at:

\bigskip

~~~~~~~~\url{\mipsurl}

\bigskip

\mipslink{} is based on code written in C language~\cite{tspopf} by Hongye Wang as a graduate student at Cornell University for optimal power flow applications~\cite{wang2007a,wang2007}. It was later ported to the \matlab{} language by Ray~D.~Zimmerman of \pserc{}\footnote{\url{http://pserc.org/}} at Cornell University for use in \matpowerlink{}~\cite{zimmerman2011,matpower}.

Up until version 6 of \matpower{}, \mips{} was distributed only as an integrated part of \matpower{}. After the release of \matpower{} 6, \mips{} was split out into a separate project, though it is still included with \matpower{} as its default AC optimal power flow solver.


\clearpage
\subsection{License and Terms of Use}

The code in \mips{} is distributed under the 3-clause BSD license\footnote{Versions 1.0 through 1.1 of \mips{} were distributed under version 3.0 of the GNU General Public License (GPL)~\cite{gpl} with an exception added to clarify our intention to allow \mips{} to interface with \matlab{} as well as any other \matlab{} code or MEX-files a user may have installed, regardless of their licensing terms. The full text of the GPL can be found at \url{https://www.gnu.org/licenses/gpl-3.0.txt}.}
~\cite{bsd}. The full text of the license can be found in the \code{LICENSE} file at the top level of the distribution or at \url{https://github.com/MATPOWER/mips/blob/master/LICENSE} and reads as follows.

\begin{Notice}
Copyright (c) 1996-2016, Power Systems Engineering Research Center
(PSERC) and individual contributors (see AUTHORS file for details).
All rights reserved.

Redistribution and use in source and binary forms, with or without
modification, are permitted provided that the following conditions
are met:

1. Redistributions of source code must retain the above copyright
notice, this list of conditions and the following disclaimer.

2. Redistributions in binary form must reproduce the above copyright
notice, this list of conditions and the following disclaimer in the
documentation and/or other materials provided with the distribution.

3. Neither the name of the copyright holder nor the names of its
contributors may be used to endorse or promote products derived from
this software without specific prior written permission.

THIS SOFTWARE IS PROVIDED BY THE COPYRIGHT HOLDERS AND CONTRIBUTORS
"AS IS" AND ANY EXPRESS OR IMPLIED WARRANTIES, INCLUDING, BUT NOT
LIMITED TO, THE IMPLIED WARRANTIES OF MERCHANTABILITY AND FITNESS
FOR A PARTICULAR PURPOSE ARE DISCLAIMED. IN NO EVENT SHALL THE
COPYRIGHT HOLDER OR CONTRIBUTORS BE LIABLE FOR ANY DIRECT, INDIRECT,
INCIDENTAL, SPECIAL, EXEMPLARY, OR CONSEQUENTIAL DAMAGES (INCLUDING,
BUT NOT LIMITED TO, PROCUREMENT OF SUBSTITUTE GOODS OR SERVICES;
LOSS OF USE, DATA, OR PROFITS; OR BUSINESS INTERRUPTION) HOWEVER
CAUSED AND ON ANY THEORY OF LIABILITY, WHETHER IN CONTRACT, STRICT
LIABILITY, OR TORT (INCLUDING NEGLIGENCE OR OTHERWISE) ARISING IN
ANY WAY OUT OF THE USE OF THIS SOFTWARE, EVEN IF ADVISED OF THE
POSSIBILITY OF SUCH DAMAGE.
\end{Notice}

\subsection{Citing \mips{}}

We request that publications derived from the use of \mips{} explicitly acknowledge that fact by citing the following 2007 paper~\cite{wang2007a}.

\begin{quote}
\footnotesize
H.~Wang, C.~E. Murillo-S{\'a}nchez, R.~D. Zimmerman, and R.~J. Thomas, ``On
Computational Issues of Market-Based Optimal Power Flow,'' \emph{Power
Systems, IEEE Transactions on}, vol.~22, no.~3, pp. 1185--1193, August 2007.\\
\doi{10.1109/TPWRS.2007.901301}
\end{quote}

The \mipsname{} (\mips{}) User's Manual~\cite{mips_manual} should also be cited explicitly
in work that refers to or is derived from its content. The citation and
DOI can be version-specific or general, as appropriate. For version 1.4,
use:

\begin{quote}
\footnotesize
R.~D. Zimmerman, H.~Wang. \mipsname{} (\mips{}) User's Manual, Verision 1.4. 2020. [Online]. Available: \url{https://matpower.org/docs/MIPS-manual-1.4.pdf}\\
\doi{10.5281/zenodo.4073324}
\end{quote}
For a version non-specific citation, use the following citation and DOI,
with \emph{\textless{}YEAR\textgreater{}} replaced by the year of the most recent release:

\begin{quote}
\footnotesize
R.~D. Zimmerman, H.~Wang. \mipsname{} (\mips{}) User's Manual. \emph{\textless{}YEAR\textgreater{}}.
[Online]. Available: \url{https://matpower.org/docs/MIPS-manual.pdf}\\
\doi{10.5281/zenodo.3236506}
\end{quote}
A list of versions of the User's Manual with release dates and
version-specific DOI's can be found via the general DOI at
\url{https://doi.org/10.5281/zenodo.3236506}.

\subsection{\mips{} Development}
\label{sec:development}

Following the release of \mips{} 1.2.2 (with \matpower{} 6.0), the \mips{} project moved to an open development paradigm, hosted on the \mips{} GitHub project page:

\bigskip

~~~~~~~~\url{\mipsurl}

\bigskip

The \mips{} GitHub project hosts the public Git code repository as well as a public issue tracker for handling bug reports, patches, and other issues and contributions. There are separate GitHub hosted repositories and issue trackers for \matpower{}, \most{}, \mpom{} and the testing framework used by all of them, \mptest{}, all available from \url{https://github.com/MATPOWER/}.


%%------------------------------------------
%\clearpage
\section{Getting Started}

\subsection{System Requirements}
\label{sec:sysreq}
To use \mips{} \mipsver{} you will need:
\begin{itemize}
\item \matlab{}\textsuperscript{\tiny \textregistered} version 7 (R14) or later\footnote{\matlab{} is available from The MathWorks, Inc. (\url{https://www.mathworks.com/}). \matlab{} is a registered trademark of The MathWorks, Inc.}, or
\item GNU Octave version 3.4 or later\footnote{GNU Octave \cite{octave} is free software, available online at \url{https://www.gnu.org/software/octave/}. \mips{} \mipsver{} may work on earlier versions of Octave, but it has not been tested on versions prior to 3.4.}
\item \mptestlink{} version 7.1 or later.\footnote{\mptest{} is available at \url{\mptesturl}.}
\end{itemize}

For the hardware requirements, please refer to the system requirements for the version of \matlab{}\footnote{\url{https://www.mathworks.com/support/sysreq/previous_releases.html}} or Octave that you are using.

In this manual, references to \matlab{} usually apply to Octave as well.

%\clearpage
\subsection{Installation}
\label{sec:installation}

{\bf Note to \matpower{} users:} \emph{\mips{} and its prerequisite, \mptest{}, are included when you install \matpower{}. There is generally no need to install them separately. You can skip directly to step 3 to verify.}

~

Installation and use of \mips{} requires familiarity with the basic operation of \matlab{} or Octave, including setting up your \matlab{}/Octave path.

\begin{enumerate}[\bfseries Step 1:] % requires package paralist
\item Clone the repository or download and extract the zip file of the \mips{} distribution from the \href{\mipsurl{}}{\mips{} project page}\footnote{\url{\mipsurl}} to the location of your choice. The files in the resulting \code{mips} or \code{mipsXXX} directory, where \code{XXX} depends on the version of \mips{}, should not need to be modified, so it is recommended that they be kept separate from your own code.
We will use \mipspath{} to denote the path to this directory.
\clearpage
\item Add the following directories to your \matlab{} or Octave path:
\begin{itemize}
\item \mipspath{/lib} -- core \mips{} functions
\item \mipspath{/lib/t} -- test scripts for \mips{}
\end{itemize}
\item At the \matlab{}/Octave prompt, type \code{test\_mips} to run the test suite and verify that \mips{} is properly installed and functioning.\footnote{The tests require a functioning installation of \mptestlink{}.} The result should resemble the following:
%\clearpage
%--LATER--
%\\\hl{re-do before release (run on fastest machine available)}
\begin{Code}
>> test_mips
t_mplinsolve......ok (6 of 66 skipped)
t_mips............ok
t_mips_pardiso....ok (60 of 60 skipped)
t_qps_mips........ok
All tests successful (192 passed, 66 skipped of 258)
Elapsed time 0.05 seconds.
\end{Code}
\end{enumerate}


\subsection{Documentation}
\label{sec:documentation}

There are two primary sources of documentation for \mips{}. The first is \href{\mipsmanurl}{this manual}, which gives an overview of the capabilities and structure of \mips{} and describes the formulations behind the code. It can be found in your \mips{} distribution at \mipspath{/docs/MIPS-manual.pdf} and the \href{\currentmipsmanurl}{latest version} is always available at: \url{\currentmipsmanurl}.

And second is the built-in \code{help} command. As with the built-in functions and toolbox routines in \matlab{} and Octave, you can type \code{help} followed by the name of a command or M-file to get help on that particular function. All of the M-files in \mips{} have such documentation and this should be considered the main reference for the calling options for each function. See Appendix~\ref{app:functions} for a list of \mips{} functions.

%%------------------------------------------
\clearpage
\section{\mips{} -- \mipsname{}}
\label{sec:mips}

\mips{}, that is, the \mipsname{}, is a primal-dual interior point solver implemented in pure \matlab{} code, derived from the MEX implementation of the algorithms included in TSPOPF~\cite{tspopf} and described in~\cite{wang2007a, wang2007}.

This solver has application to general nonlinear optimization problems of the following form:
\begin{equation}
\min_x f(x) \label{eq:mips_prob_begin}
\end{equation}
subject to
\begin{eqnarray}
& g(x) = 0 & \label{eq:mips_g}  \\
& h(x) \le 0 & \label{eq:mips_h}  \\
& l \le A x \le u  & \label{eq:mips_linear_constraints}  \\
& x_\mathrm{min} \le x \le x_\mathrm{max} & \label{eq:mips_var_bounds}
\end{eqnarray}
where $f \colon \R^n \to \R$, $g \colon \R^n \to \R^m$ and $h \colon \R^n \to \R^p$.

The solver is implemented by the \code{mips} function, which can be called as follows,
\begin{Code}
[x, f, exitflag, output, lambda] = ...
    mips(f_fcn, x0, A, l, u, xmin, xmax, gh_fcn, hess_fcn, opt);
\end{Code}
where the input and output arguments are described in Tables~\ref{tab:mips_input} and \ref{tab:mips_output}, respectively. Alternatively, the input arguments can be packaged as fields in a \code{problem} struct and passed in as a single argument, where all fields except \code{f\_fcn} and \code{x0} are optional.
\begin{Code}
[x, f, exitflag, output, lambda] = mips(problem);
\end{Code}


\begin{table}[!ht]
%\renewcommand{\arraystretch}{1.2}
\centering
\begin{threeparttable}
\caption{Input Arguments for \code{mips}\tnote{\dag}}
\label{tab:mips_input}
\footnotesize
\begin{tabular}{lp{0.85\textwidth}}
\toprule
name & description \\
\midrule
\code{f\_fcn}	& Handle to a function that evaluates the objective function, its gradients and Hessian\tnote{\ddag} for a given value of $x$. Calling syntax for this function:	\\
&~~~~\code{[f, df, d2f] = f\_fcn(x)}	\\
\code{x0}	& Starting value of optimization vector $x$.	\\
\code{A}, \code{l}, \code{u}	& Define the optional linear constraints $l \le A x \le u$. Default values for the elements of \code{l} and \code{u} are \code{-Inf} and \code{Inf}, respectively.	\\
\code{xmin}, \code{xmax}	& Optional lower and upper bounds on the $x$ variables, defaults are \code{-Inf} and \code{Inf}, respectively.	\\
\code{gh\_fcn}	& Handle to function that evaluates the optional nonlinear constraints and their gradients for a given value of $x$. Calling syntax for this function is:	\\
&~~~~\code{[h, g, dh, dg] = gh\_fcn(x)}	\\
& where the columns of \code{dh} and \code{dg} are the gradients of the corresponding elements of \code{h} and \code{g}, i.e. \code{dh} and \code{dg} are transposes of the Jacobians of \code{h} and \code{g}, respectively.	\\
\code{hess\_fcn}	& Handle to function that computes the Hessian\tnote{\ddag} of the Lagrangian for given values of $x$, $\lambda$ and $\mu$, where $\lambda$ and $\mu$ are the multipliers on the equality and inequality constraints, $g$ and $h$, respectively. The calling syntax for this function is:	\\
&~~~~\code{Lxx = hess\_fcn(x, lam, cost\_mult)},	\\
&where $\lambda$ = \code{lam.eqnonlin}, $\mu$ = \code{lam.ineqnonlin} and \code{cost\_mult} is a parameter used to scale the objective function	\\
\code{opt}	& Optional options structure with fields, all of which are also optional, described in Table~\ref{tab:mips_options}. \\
\code{problem}	& Alternative, single argument input struct with fields corresponding to arguments above.	\\
\bottomrule
\end{tabular}
\begin{tablenotes}
 \scriptsize
 \item [\dag] {All inputs are optional except \code{f\_fcn} and \code{x0}.}
 \item [\ddag] {If \code{gh\_fcn} is provided then \code{hess\_fcn} is also required. Specifically, if there are nonlinear constraints, the Hessian information must be provided by the \code{hess\_fcn} function and it need not be computed in \code{f\_fcn}.}
\end{tablenotes}
\end{threeparttable}
\end{table}


\begin{table}[!ht]
%\renewcommand{\arraystretch}{1.2}
\centering
\begin{threeparttable}
\caption{Output Arguments for \code{mips}}
\label{tab:mips_output}
\footnotesize
\begin{tabular}{lp{0.8\textwidth}}
\toprule
name & description \\
\midrule
\code{x}	& solution vector	\\
\code{f}	& final objective function value	\\
\code{exitflag}	& exit flag	\\
&\begin{tabular}{r @{ -- } l}
1 & first order optimality conditions satisfied \\
0 & maximum number of iterations reached \\
-1 & numerically failed \\
\end{tabular}	\\
\code{output}	& output struct with fields	\\
&\begin{tabular}{lp{0.65\textwidth}}
\code{iterations} & number of iterations performed \\
\code{hist} & struct array with trajectories of the following: \code{feascond}, \code{gradcond}, \code{compcond}, \code{costcond}, \code{gamma}, \code{stepsize}, \code{obj}, \code{alphap}, \code{alphad} \\
\code{message} & exit message \\
\end{tabular}	\\
\code{lambda}	& struct containing the Langrange and Kuhn-Tucker multipliers on the constraints, with fields:	\\
&\begin{tabular}{lp{0.65\textwidth}}
\code{eqnonlin} & nonlinear equality constraints	\\
\code{ineqnonlin} & nonlinear inequality constraints	\\
\code{mu\_l} & lower (left-hand) limit on linear constraints	\\
\code{mu\_u} & upper (right-hand) limit on linear constraints	\\
\code{lower} & lower bound on optimization variables	\\
\code{upper} & upper bound on optimization variables	\\
\end{tabular}	\\
\bottomrule
\end{tabular}
\end{threeparttable}
\end{table}

\begin{table}[!ht]
%\renewcommand{\arraystretch}{1.2}
\centering
\begin{threeparttable}
\caption{Options for \code{mips}}
\label{tab:mips_options}
\footnotesize
\begin{tabular}{lcp{0.65\textwidth}}
\toprule
name & default & description \\
\midrule
\code{opt.verbose} & 0 & controls level of progress output displayed \\
&& \begin{tabular}{r @{ -- } l}
0 & print no progress info \\
1 & print a little progress info \\
2 & print a lot of progress info \\
3 & print all progress info \\
\end{tabular}	\\
\code{opt.linsolver} & \codeq{} & linear system solver for solving update steps \eqref{eq:ipm_reduced_system}, i.e. \code{solver} input to \code{mplinsolve} \\
&& \begin{tabular}{r @{ -- } p{0.5\textwidth}}
\codeq{} & default, same as \codeq{\symbol{92}} \\
\codeq{\symbol{92}} & use built-in \code{\symbol{92}} operator \\
\codeq{LU} & use \code{lu} to explicitly factor matrix, followed by back-substitution \\
\codeq{PARDISO} & use optional third-party \pardiso{} solver\tnote{\dag} \\
\end{tabular}	\\
\code{opt.feastol} & $10^{-6}$ & termination tolerance for feasibility condition \\
\code{opt.gradtol} & $10^{-6}$ & termination tolerance for gradient condition \\
\code{opt.comptol} & $10^{-6}$ & termination tolerance for complementarity condition \\
\code{opt.costtol} & $10^{-6}$ & termination tolerance for cost condition \\
\code{opt.max\_it} & 150 & maximum number of iterations \\
\code{opt.step\_control} & 0 & set to 1 to enable step-size control \\
\code{opt.sc.red\_it} & 20 & max number of step-size reductions if step-control is on \\
\code{opt.cost\_mult} & 1 & cost multiplier used to scale the objective function for improved conditioning. Note: This value is also passed as the 3\textsuperscript{rd} argument to the Hessian evaluation function so that it can appropriately scale the objective function term in the Hessian of the Lagrangian.	\\
\code{opt.xi} & 0.99995 & $\xi$ constant used in $\alpha$ updates in \eqref{eq:alphap} and \eqref{eq:alphad} \\
\code{opt.sigma} & 0.1 & centering parameter $\sigma$ used in $\gamma$ update in \eqref{eq:gamma} \\
\code{opt.z0} & 1 & used to initialize elements of slack variable $Z$ \\
\code{opt.alpha\_min} & $10^{-8}$ & algorithm returns ``Numerically Failed'' if the $\alpha_p$ or $\alpha_d$ from \eqref{eq:alphap} and \eqref{eq:alphad} become smaller than this value \\
\code{opt.rho\_min} & 0.95 & lower bound on $\rho_t$ corresponding to $1 - \eta$ in Fig.~5 in \cite{wang2007a} \\
\code{opt.rho\_max} & 1.05 & upper bound on $\rho_t$ corresponding to $1 + \eta$ in Fig.~5 in \cite{wang2007a} \\
\code{opt.mu\_threshold} & $10^{-5}$ & Kuhn-Tucker multipliers smaller than this value for non-binding constraints are forced to zero \\
\code{opt.max\_stepsize} & $10^{10}$ & algorithm returns ``Numerically Failed'' if the 2-norm of the Newton step $\left[\begin{array}{c}\Delta X \\ \Delta \lambda \\ \end{array}\right]$ from \eqref{eq:ipm_reduced_system} exceeds this value \\
\bottomrule
\end{tabular}
\begin{tablenotes}
 \scriptsize
 \item [\dag] {Requires the installation of an optional package. See Appendix~\ref{app:pardiso} for details.}
\end{tablenotes}
\end{threeparttable}
\end{table}


The calling syntax is nearly identical to that used by \code{fmincon} from \matlab{}'s \ot{}. The primary difference is that the linear constraints are specified in terms of a single doubly-bounded linear function ($l \le A x \le u$) as opposed to separate equality constrained ($A_{eq} x = b_{eq}$) and upper bounded ($A x \le b$) functions. Internally, equality constraints are handled explicitly and determined at run-time based on the values of $l$ and $u$.

The user-defined functions for evaluating the objective function, constraints and Hessian are identical to those required by \code{fmincon}, with one exception described below for the Hessian evaluation function. Specifically, \code{f\_fcn} should return \code{f} as the scalar objective function value $f(x)$, \code{df} as an $n \times 1$ vector equal to $\nabla f$ and, unless \code{gh\_fcn} is provided and the Hessian is computed by \code{hess\_fcn}, \code{d2f} as an $n \times n$ matrix equal to the Hessian $\der{^2f}{x^2}$. Similarly, the constraint evaluation function \code{gh\_fcn} must return the $m \times 1$ vector of nonlinear equality constraint violations $g(x)$, the $p \times 1$ vector of nonlinear inequality constraint violations $h(x)$ along with their gradients in \code{dg} and \code{dh}. Here \code{dg} is an $n \times m$ matrix whose $j^\mathrm{th}$ column is $\nabla g_j$ and \code{dh} is $n \times p$, with $j^\mathrm{th}$ column equal to $\nabla h_j$. Finally, for cases with nonlinear constraints, \code{hess\_fcn} returns the $n \times n$ Hessian $\der{^2\mathcal{L}}{x^2}$ of the Lagrangian function
\begin{equation}
\mathcal{L}(x, \lambda, \mu, \sigma) = \sigma f(x) + \trans{\lambda} g(x) + \trans{\mu} h(x)
\end{equation}
for given values of the multipliers $\lambda$ and $\mu$, where $\sigma$ is the \code{cost\_mult} scale factor for the objective function. Unlike \code{fmincon}, \code{mips} passes this scale factor to the Hessian evaluation function in the 3\textsuperscript{rd} input argument.

The use of \code{nargout} in \code{f\_fcn} and \code{gh\_fcn} is recommended so that the gradients and Hessian are only computed when required.

\clearpage
\subsection{Example 1}
\label{sec:ex1}

The following code, included as \code{mips\_example1.m} in \mipspath{lib/t}, shows a simple example of using \code{mips} to solve a 2-dimensional unconstrained optimization of Rosenbrock's ``banana'' function\footnote{\url{https://en.wikipedia.org/wiki/Rosenbrock_function}}
\begin{equation}
f(x) = 100 (x_2-x_1^2)^2+(1-x_1)^2.
\end{equation}

First, create a \matlab{} function that will evaluate the objective function, its gradients and Hessian, for a given value of $x$. In this case, the coefficient of the first term is defined as a paramter \code{a}.
\begin{Code}
function [f, df, d2f] = banana(x, a)
f = a*(x(2)-x(1)^2)^2+(1-x(1))^2;
if nargout > 1          %% gradient is required
    df = [  4*a*(x(1)^3 - x(1)*x(2)) + 2*x(1)-2;
            2*a*(x(2) - x(1)^2)                     ];
    if nargout > 2      %% Hessian is required
        d2f = 4*a*[ 3*x(1)^2 - x(2) + 1/(2*a),  -x(1);
                    -x(1)                       1/2 ];
    end
end
\end{Code}
Then, create a handle to the function, defining the value of the paramter \code{a} to be 100, set up the starting value of $x$, and call the \code{mips} function to solve it.
\begin{Code}
>> f_fcn = @(x)banana(x, 100);
>> x0 = [-1.9; 2];
>> [x, f] = mips(f_fcn, x0)

x =

     1
     1


f =

     0

\end{Code}


\subsection{Example 2}
\label{sec:ex2}

The second example\footnote{From \url{https://en.wikipedia.org/wiki/Nonlinear\_programming\#3-dimensional\_example}.} solves the following 3-dimensional constrained optimization, printing the details of the solver's progress:
\begin{equation}
\min_x f(x) = -x_1 x_2 - x_2 x_3
\end{equation}
subject to
\begin{eqnarray}
x_1^2 - x_2^2 + x_3^2 - 2 & \le & 0 \\
x_1^2 + x_2^2 + x_3^2 - 10 & \le & 0.
\end{eqnarray}

First, create a \matlab{} function to evaluate the objective function and its gradients,\footnote{Since the problem has nonlinear constraints and the Hessian is provided by \code{hess\_fcn}, this function will never be called with three output arguments, so the code to compute \code{d2f} is actually not necessary.}
\begin{Code}
function [f, df, d2f] = f2(x)
f = -x(1)*x(2) - x(2)*x(3);
if nargout > 1           %% gradient is required
    df = -[x(2); x(1)+x(3); x(2)];
    if nargout > 2       %% Hessian is required
        d2f = -[0 1 0; 1 0 1; 0 1 0];   %% actually not used since
    end                                 %% 'hess_fcn' is provided
end
\end{Code}
one to evaluate the constraints, in this case inequalities only, and their gradients,
\begin{Code}
function [h, g, dh, dg] = gh2(x)
h = [ 1 -1 1; 1 1 1] * x.^2 + [-2; -10];
dh = 2 * [x(1) x(1); -x(2) x(2); x(3) x(3)];
g = []; dg = [];
\end{Code}
and another to evaluate the Hessian of the Lagrangian.
\begin{Code}
function Lxx = hess2(x, lam, cost_mult)
if nargin < 3, cost_mult = 1; end   %% allows to be used with 'fmincon'
mu = lam.ineqnonlin;
Lxx = cost_mult * [0 -1 0; -1 0 -1; 0 -1 0] + ...
        [2*[1 1]*mu 0 0; 0 2*[-1 1]*mu 0; 0 0 2*[1 1]*mu];
\end{Code}
Then create a \code{problem} struct with handles to these functions, a starting value for $x$ and an option to print the solver's progress. Finally, pass this struct to \code{mips} to solve the problem and print some of the return values to get the output below.
\begin{Code}
function mips_example2
problem = struct( ...
    'f_fcn',    @(x)f2(x), ...
    'gh_fcn',   @(x)gh2(x), ...
    'hess_fcn', @(x, lam, cost_mult)hess2(x, lam, cost_mult), ...
    'x0',       [1; 1; 0], ...
    'opt',      struct('verbose', 2) ...
);
[x, f, exitflag, output, lambda] = mips(problem);
fprintf('\nf = %g   exitflag = %d\n', f, exitflag);
fprintf('\nx = \n');
fprintf('   %g\n', x);
fprintf('\nlambda.ineqnonlin =\n');
fprintf('   %g\n', lambda.ineqnonlin);
\end{Code}
\begin{Code}
>> mips_example2
MATPOWER Interior Point Solver -- MIPS, Version 1.4, 08-Oct-2020
 (using built-in linear solver)
 it    objective   step size   feascond     gradcond     compcond     costcond  
----  ------------ --------- ------------ ------------ ------------ ------------
  0            -1                       0          1.5            5            0
  1    -5.3250167     1.6875            0     0.894235     0.850653      2.16251
  2    -7.4708991    0.97413     0.129183   0.00936418     0.117278     0.339269
  3    -7.0553031    0.10406            0   0.00174933    0.0196518    0.0490616
  4    -7.0686267   0.034574            0   0.00041301    0.0030084   0.00165402
  5    -7.0706104  0.0065191            0  1.53531e-05  0.000337971  0.000245844
  6    -7.0710134 0.00062152            0  1.22094e-07  3.41308e-05  4.99387e-05
  7    -7.0710623 5.7217e-05            0  9.84879e-10  3.41587e-06  6.05875e-06
  8    -7.0710673 5.6761e-06            0  9.73527e-12  3.41615e-07  6.15483e-07
Converged!

f = -7.07107   exitflag = 1

x = 
   1.58114
   2.23607
   1.58114

lambda.ineqnonlin =
   0
   0.707107
\end{Code}
This example can be found in \code{mips\_example2.m}. More example problems for \code{mips} can be found in \code{t\_mips.m}, both in \mipspath{lib/t}.

\subsection{Primal-Dual Interior Point Algorithm}

This section provides some details on the primal-dual interior point algorithm used by \mips{} and described in~\cite{wang2007a, wang2007}.

\subsubsection{Notation}

For a scalar function $f \colon \R^n \to \R$ of a real vector $X = \trans{\left[\begin{array}{cccc}x_1 & x_2 & \cdots & x_n\end{array}\right]}$, we use the following notation for the first derivatives (transpose of the gradient):
\begin{equation}
f_X = \der{f}{X} = \left[\begin{array}{cccc}
\der{f}{x_1} &
\der{f}{x_2} &
\cdots &
\der{f}{x_n}
\end{array}\right].
\end{equation}
The matrix of second partial derivatives, the Hessian of $f$, is:
\begin{equation}
f_{XX} = \der{^2f}{X^2} = \der{}{X}\trans{\left( \der{f}{X} \right)}
= \left[\begin{array}{cccc}
\der{^2f}{x_1^2} &
\cdots &
\der{^2f}{x_1 \partial x_n} \\
\vdots &
\ddots &
\vdots \\
\der{^2f}{x_n \partial x_1} &
\cdots &
\der{^2f}{x_n^2} \\
\end{array}\right].
\end{equation}

For a vector function $F \colon \R^n \to \R^m$ of a vector $X$, where
\begin{equation}
F(X) = \trans{\left[\begin{array}{cccc}
f_1(X) & f_2(X) & \cdots & f_m(X)
\end{array}\right]}
\end{equation}
the first derivatives form the Jacobian matrix, where row $i$ is the transpose of the gradient of $f_i$
\begin{equation}
F_X = \der{F}{X} = \left[\begin{array}{cccc}
\der{f_1}{x_1} &
\cdots &
\der{f_1}{x_n} \\
\vdots &
\ddots &
\vdots \\
\der{f_m}{x_1} &
\cdots &
\der{f_m}{x_n} \\
\end{array}\right].
\end{equation}
In these derivations, the full 3-dimensional set of second partial derivatives of $F$ will not be computed. Instead a matrix of partial derivatives will be formed by computing the Jacobian of the vector function obtained by multiplying the transpose of the Jacobian of $F$ by a vector $\lambda$, using the following notation
\begin{equation}
F_{XX}(\lambda) = \der{}{X} \left( \trans{F_X} \lambda \right).
\end{equation}

Please note also that $\diag{A}$ is used to denote a diagonal matrix with vector $A$ on the diagonal and $e$ is a vector of all ones.


\subsubsection{Problem Formulation and Lagrangian}

The primal-dual interior point method used by \mips{} solves a problem of the form:
\begin{equation}
\min_X f(X)
\end{equation}
subject to
\begin{align}
G(X) &= 0 \\
H(X) &\le 0
\end{align}
where the linear constraints and variable bounds from \eqref{eq:mips_linear_constraints} and \eqref{eq:mips_var_bounds} have been incorporated into $G(X)$ and $H(X)$. The approach taken involves converting the $n_i$ inequality constraints into equality constraints using a barrier function and vector of positive slack variables $Z$.
\begin{equation}
\min_X \left[f(X) - \gamma \sum_{m=1}^{n_i} \ln(Z_m)\right]
\end{equation}
subject to
\begin{align}
G(X) &= 0 \\
H(X) + Z &= 0 \\
Z &> 0
\end{align}
As the parameter of perturbation $\gamma$ approaches zero, the solution to this problem approaches that of the original problem.

For a given value of $\gamma$, the Lagrangian for this equality constrained problem is
\begin{equation}
\mathcal{L}^\gamma(X, Z, \lambda, \mu) = f(X) + \trans{\lambda} G(X) + \trans{\mu} (H(X) + Z) - \gamma \sum_{m=1}^{n_i} \ln(Z_m).
\label{eq:L}
\end{equation}
Taking the partial derivatives with respect to each of the variables yields:
\begin{eqnarray}
\mathcal{L}^\gamma_X(X, Z, \lambda, \mu) &=& f_X + \trans{\lambda} G_X + \trans{\mu} H_X \\
\mathcal{L}^\gamma_Z(X, Z, \lambda, \mu) &=& \trans{\mu} - \gamma \trans{e} \diag{Z}^{-1} \\
\mathcal{L}^\gamma_\lambda(X, Z, \lambda, \mu) &=& \trans{G}(X) \\
\mathcal{L}^\gamma_\mu(X, Z, \lambda, \mu) &=& \trans{H}(X) + \trans{Z}.
\end{eqnarray}
And the Hessian of the Lagrangian with respect to $X$ is given by
\begin{equation}
\mathcal{L}^\gamma_{XX}(X, Z, \lambda, \mu) = f_{XX} + G_{XX}(\lambda) + H_{XX}(\mu).
\end{equation}

\subsubsection{First Order Optimality Conditions}

The first order optimality (Karush-Kuhn-Tucker) conditions for this problem are satisfied when the partial derivatives of the Lagrangian above are all set to zero:
\begin{eqnarray}
F(X, Z, \lambda, \mu) = 0 && \\
Z > 0 && \\
\mu > 0 &&
\end{eqnarray}
where
\begin{equation}
F(X, Z, \lambda, \mu) = \left[\begin{array}{c}
\trans{\mathcal{L}^\gamma_X} \\
\diag{\mu} Z - \gamma e \\
G(X) \\
H(X) + Z \\
\end{array}\right] = \left[\begin{array}{c}
\trans{f_X} + \trans{G_X} \lambda + \trans{H_X} \mu \\
\diag{\mu} Z - \gamma e \\
G(X) \\
H(X) + Z \\
\end{array}\right].
\end{equation}

\subsubsection{Newton Step}

The first order optimality conditions are solved using Newton's method. The Newton update step can be written as follows:
\begin{equation}
\left[\begin{array}{cccc}
F_X & F_Z & F_\lambda & F_\mu
\end{array}\right]
\left[\begin{array}{c}
\Delta X \\
\Delta Z \\
\Delta \lambda \\
\Delta \mu
\end{array}\right]
= -F(X, Z, \lambda, \mu)
\end{equation}
\begin{equation}
\left[\begin{array}{cccc}
\mathcal{L}^\gamma_{XX} & 0 & \trans{G_X} & \trans{H_X} \\
0 & \diag{\mu} & 0 & \diag{Z} \\
G_X & 0 & 0 & 0 \\
H_X & I & 0 & 0
\end{array}\right]
\left[\begin{array}{c}
\Delta X \\
\Delta Z \\
\Delta \lambda \\
\Delta \mu
\end{array}\right]
= -\left[\begin{array}{c}
\trans{\mathcal{L}^\gamma_X} \\
\diag{\mu} Z - \gamma e \\
G(X) \\
H(X) + Z \\
\end{array}\right].
\label{eq:newton_step}
\end{equation}

This set of equations can be simplified and reduced to a smaller set of equations by solving explicitly for $\Delta \mu$ in terms of $\Delta Z$ and for $\Delta Z$ in terms of $\Delta X$. Taking the 2\textsuperscript{nd} row of \eqref{eq:newton_step} and solving for $\Delta \mu$ we get
\begin{align}
\diag{\mu} \Delta Z + \diag{Z} \Delta \mu &= -\diag{\mu} Z + \gamma e \nonumber \\
\diag{Z} \Delta \mu &= -\diag{Z} \mu + \gamma e - \diag{\mu} \Delta Z \nonumber \\
\Delta \mu &= - \mu + \diag{Z}^{-1} (\gamma e - \diag{\mu} \Delta Z).
\label{eq:2nd_row}
\end{align}
Solving the 4\textsuperscript{th} row of \eqref{eq:newton_step} for $\Delta Z$ yields
\begin{align}
H_X \Delta X + \Delta Z &= -H(X) - Z \nonumber \\
\Delta Z &= -H(X) - Z - H_X \Delta X.
\label{eq:4th_row}
\end{align}
Then, substituting \eqref{eq:2nd_row} and \eqref{eq:4th_row} into the 1\textsuperscript{st} row of \eqref{eq:newton_step} results in
\begin{alignat}{2}
&\mathcal{L}^\gamma_{XX} \Delta X + \trans{G_X} \Delta \lambda + \trans{H_X} \Delta \mu &&\;= -\trans{\mathcal{L}^\gamma_X} \nonumber \\
&\mathcal{L}^\gamma_{XX} \Delta X + \trans{G_X} \Delta \lambda + \trans{H_X} (- \mu + \diag{Z}^{-1} (\gamma e - \diag{\mu} \Delta Z)) &&\;= -\trans{\mathcal{L}^\gamma_X} \nonumber \\
&\mathcal{L}^\gamma_{XX} \Delta X + \trans{G_X} \Delta \lambda & \nonumber \\
&+ \trans{H_X} (- \mu + \diag{Z}^{-1} (\gamma e - \diag{\mu} (-H(X) - Z - H_X \Delta X))) &&\;= -\trans{\mathcal{L}^\gamma_X} \nonumber \\
&\mathcal{L}^\gamma_{XX} \Delta X + \trans{G_X} \Delta \lambda - \trans{H_X} \mu + \trans{H_X} \diag{Z}^{-1} \gamma e & \nonumber \\
&+ \trans{H_X} \diag{Z}^{-1} \diag{\mu} H(X) + \trans{H_X} \diag{Z}^{-1} \diag{Z} \mu + \trans{H_X} \diag{Z}^{-1} \diag{\mu} H_X \Delta X &&\;= -\trans{\mathcal{L}^\gamma_X} \nonumber \\
&(\mathcal{L}^\gamma_{XX} + \trans{H_X} \diag{Z}^{-1} \diag{\mu} H_X) \Delta X + \trans{G_X} \Delta \lambda  & \nonumber \\
&+ \trans{H_X} \diag{Z}^{-1} (\gamma e + \diag{\mu} H(X)) &&\;= -\trans{\mathcal{L}^\gamma_X} \nonumber \\
&M \Delta X + \trans{G_X} \Delta \lambda &&\;= -N
\label{eq:1st_row}
\end{alignat}
where
\begin{align}
M &\equiv \mathcal{L}^\gamma_{XX} + \trans{H_X} \diag{Z}^{-1} \diag{\mu} H_X  \\
  &= f_{XX} + G_{XX}(\lambda) + H_{XX}(\mu) + \trans{H_X} \diag{Z}^{-1} \diag{\mu} H_X
\end{align}
and
\begin{align}
N &\equiv \trans{\mathcal{L}^\gamma_X} + \trans{H_X} \diag{Z}^{-1} (\gamma e + \diag{\mu} H(X)) \\
  &= \trans{f_X} + \trans{G_X} \lambda + \trans{H_X} \mu + \trans{H_X} \diag{Z}^{-1} (\gamma e + \diag{\mu} H(X)).
\end{align}

Combining \eqref{eq:1st_row} and the 3\textsuperscript{rd} row of \eqref{eq:newton_step} results in a system of equations of reduced size:
\begin{equation}
\left[\begin{array}{cc}
M & \trans{G_X} \\
G_X & 0 \\
\end{array}\right]
\left[\begin{array}{c}
\Delta X \\
\Delta \lambda \\
\end{array}\right]
= \left[\begin{array}{c}
-N \\
-G(X) \\
\end{array}\right]. \label{eq:ipm_reduced_system}
\end{equation}
The Newton update can then be computed in the following 3 steps:

\begin{enumerate}
\item Compute $\Delta X$ and $\Delta \lambda$ from \eqref{eq:ipm_reduced_system}.
\item Compute $\Delta Z$ from \eqref{eq:4th_row}.
\item Compute $\Delta \mu$ from \eqref{eq:2nd_row}.
\end{enumerate}

In order to maintain strict feasibility of the trial solution, the algorithm truncates the Newton step by scaling the primal and dual variables by $\alpha_p$ and $\alpha_d$, respectively, where these scale factors are computed as follows:
\begin{align}
\alpha_p &= \min \left(\xi \min_{\Delta Z_m < 0} \left(-\frac{Z_m}{\Delta Z_m}\right), 1 \right)  \label{eq:alphap} \\
\alpha_d &= \min \left(\xi \min_{\Delta \mu_m < 0} \left(-\frac{\mu_m}{\Delta \mu_m}\right), 1 \right)  \label{eq:alphad}
\end{align}
resulting in the variable updates below.
\begin{align}
X &\gets X + \alpha_p \Delta X  \\
Z &\gets Z + \alpha_p \Delta Z  \\
\lambda &\gets \lambda + \alpha_d \Delta \lambda  \\
\mu &\gets \mu + \alpha_d \Delta \mu
\end{align}

The parameter $\xi$ is a constant scalar with a value slightly less than one. In \mips{}, $\xi$ is set to 0.99995.

In this method, during the Newton-like iterations, the perturbation parameter $\gamma$ must converge to zero in order to satisfy the first order optimality conditions of the original problem. \mips{} uses the following rule to update $\gamma$ at each iteration, after updating $Z$ and $\mu$:
\begin{equation}
\gamma \gets \sigma \frac{\trans{Z} \mu}{n_i} \label{eq:gamma}
\end{equation}
where $\sigma$ is a scalar constant between 0 and 1. In \mips{}, $\sigma$ is set to 0.1.

%%------------------------------------------
\clearpage
\section{Other Functions}

\subsection{Linear System Solver - {\tt mplinsolve}}
\label{sec:mplinsolve}

The \code{mplinsolve} function provides a common interface to several different methods for solving a system of linear equations of the form $Ax = b$.
\begin{Code}
x = mplinsolve(A, b);
x = mplinsolve(A, b, solver);
x = mplinsolve(A, b, solver, opt);
\end{Code}

The \code{solver} argument can be one of the following:
\begin{itemize}
\setlength{\parskip}{-6pt}%
\item \codeq{} -- default, same as \codeq{\symbol{92}}
\item \codeq{\symbol{92}} -- use built-in \code{\symbol{92}} operator, i.e. \code{x = A \symbol{92} b}
\item \codeq{LU} -- use built-in \code{lu} function to explicitly factor $A$, then solve for $x$ via back-substitution
\item \codeq{PARDISO} -- use optional third-party \pardiso{} solver\footnote{Requires the installation of an optional package. See Appendix~\ref{app:pardiso} for details.}
\end{itemize}
See \code{help mplinsolve} for more details.

Note that \mips{} uses \code{mplinsolve} to solve the Newton update step in \eqref{eq:ipm_reduced_system}.

\subsection{Quadratic Programming Solver - {\tt qps\_mips}}
\label{sec:qps_mips}

A convenience wrapper function called \code{qps\_mips} is provided to make it trivial to set up and solve linear programming (LP) and quadratic programming (QP) problems of the following form:
\begin{equation}
\min_x \frac{1}{2} \trans{x} H x + \trans{c} x
\end{equation}
subject to
\begin{eqnarray}
& l \le A x \le u  & \\
& x_\mathrm{min} \le x \le x_\mathrm{max}. &
\end{eqnarray}
Instead of a function handle, the objective function is specified in terms of the paramters $H$ and $c$ of quadratic cost coefficients. Internally, \code{qps\_mips} passes \code{mips} the handle of a function that uses these paramters to evaluate the objective function, gradients and Hessian.

The calling syntax for \code{qps\_mips} is similar to that used by \code{quadprog} from the \matlab{} \ot{}.
\begin{Code}
[x, f, exitflag, output, lambda] = qps_mips(H, c, A, l, u, xmin, xmax, x0, opt);
\end{Code}
Alternatively, the input arguments can be packaged as fields in a \code{problem} struct and passed in as a single argument, where all fields except \code{H}, \code{c}, \code{A} and \code{l} are optional.
\begin{Code}
[x, f, exitflag, output, lambda] = qps_mips(problem);
\end{Code}
Aside from \code{H} and \code{c}, all input and output arguments correspond exactly to the same arguments for \code{mips} as described in Tables~\ref{tab:mips_input} and \ref{tab:mips_output}.

As with \code{mips} and \code{fmincon}, the primary difference between the calling syntax for \code{qps\_mips} and \code{quadprog} is that the linear constraints are specified in terms of a single doubly-bounded linear function ($l \le A x \le u$) as opposed to separate equality constrained ($A_{eq} x = b_{eq}$) and upper bounded ($A x \le b$) functions.

Several examples of using \code{qps\_mips} to solve LP and QP problems can be found in \code{t\_qps\_mips.m}.

\subsection{Private Feature Detection Functions}
\label{sec:featuredetection}

The following are private functions that implement detection of specific optional functionality. They are not intended to be called directly, but rather are used to extend the capabilities of \code{have\_feature}, a function included in \mptestlink{} and described in the \mptest{} \href{\mptesturl/blob/master/README.md}{\code{README}} file.

\subsubsection{\tt have\_feature\_lu\_vec}
\label{sec:have_feature_lu_vec}

This function implements the \codeq{lu\_vec} tag for \code{have\_feature} to detect support for the \code{lu(..., \textquotesingle{}vector\textquotesingle{})} syntax.

\subsubsection{\tt have\_feature\_pardiso\_legacy}
\label{sec:have_feature_pardiso_legacy}

This function implements the \codeq{pardiso\_legacy} tag for \code{have\_feature} to detect support for the legacy (v5.x) \pardiso{} interface, with individual MEX files for factor, solve, etc.

\subsubsection{\tt have\_feature\_pardiso\_object}
\label{sec:have_feature_pardiso_object}

This function implements the \codeq{pardiso\_object} tag for \code{have\_feature} to detect support for the object-oriented (v6.x and later) \pardiso{} interface.

\subsubsection{\tt have\_feature\_pardiso}
\label{sec:have_feature_pardiso}

This function implements the \codeq{pardiso} tag for \code{have\_feature} to detect availability/version of \pardiso{}, Parallel Sparse Direct and Multi-Recursive Iterative Linear Solvers, available from \url{https://pardiso-project.org}. See also Appendix~\ref{app:pardiso}.


%%------------------------------------------
\clearpage
\section{Acknowledgments}
The authors would like to acknowledge the support of the research grants and contracts that have contributed directly and indirectly to the development of \mips{}. This includes funding from the \PSERC{}, the U.S. Department of Energy,\footnote{Supported in part by the \CERTS{} and the Office of Electricity Delivery and Energy Reliability, Transmission Reliability Program of the U.S. Department of Energy under the National Energy Technology Laboratory Cooperative Agreement No.~DE-FC26-09NT43321.} and the National Science Foundation.\footnote{This material is based upon work supported in part by the National Science Foundation under Grant Nos. 0532744, 1642341 and 1931421. Any opinions, findings, and conclusions or recommendations expressed in this material are those of the author(s) and do not necessarily reflect the views of the National Science Foundation.}


\begin{appendices}

%%------------------------------------------
\clearpage
\section{\mips{} Files and Functions}
\label{app:functions}

This appendix lists all of the files and functions that \mips{} provides. In most cases, the function is found in a \matlab{} M-file in the \code{lib} directory of the distribution, where the \code{.m} extension is omitted from this listing. For more information on each, at the \matlab{}/Octave prompt, simply type \code{help} followed by the name of the function. For documentation and other files, the filename extensions are included.

\begin{table}[!ht]
%\renewcommand{\arraystretch}{1.2}
\centering
\begin{threeparttable}
\caption{\mips{} Files and Functions}
\label{tab:files}
\footnotesize
\begin{tabular}{p{0.34\textwidth}p{0.59\textwidth}}
\toprule
name & description \\
\midrule
\code{AUTHORS}	& list of authors and contributors	\\
\code{CHANGES}	& \mips{} change history	\\
\code{CITATION}	& info on how to cite \mips{}	\\
\code{CONTRIBUTING.md}	& notes on how to contribute to the \mips{} project	\\
\code{LICENSE}	& \mips{} license (3-clause BSD license)	\\
\code{README.md}	& basic introduction to \mips{}	\\
\code{docs/}	& 	\\
\code{~~MIPS-manual.pdf}	& \mipsman{}	\\
\code{~~src/MIPS-manual/}	&	\\
\code{~~~~MIPS-manual.tex}	& LaTeX source for \mips{} User's Manual	\\
\code{lib/}	& 	\\
\code{~~have\_feature\_lu\_vec}\tnote{*}	& support for \code{lu(..., \textquotesingle{}vector\textquotesingle{})} syntax	\\
\code{~~have\_feature\_pardiso\_legacy}\tnote{*}	& \pardiso{} v5, individual MEX files for factor, solve, etc.	\\
\code{~~have\_feature\_pardiso\_object}\tnote{*}	& \pardiso{} v6 and later, object interface	\\
\code{~~have\_feature\_pardiso}\tnote{*}	& \pardiso{}, {\bf Par}allel Sparse {\bf D}irect \& {\bf I}terative Linear {\bf So}lver, \url{https://pardiso-project.org}	\\
\code{~~mips}	& \mipsname{} -- primal/dual interior point solver for NLP	\\
\code{~~mipsver}	& prints version information for \mips{}	\\
\code{~~mplinsolve}	& common linear system solver interface, used by \mips{}	\\
\code{~~qps\_mips}	& common QP/LP solver interface to \mips{}-based solver	\\
\code{~~t/}	& 	\\
\code{~~~~mips\_example1}	& implements example 1 from \mips{} User's Manual	\\
\code{~~~~mips\_example2}	& implements example 2 from \mips{} User's Manual	\\
\code{~~~~test\_mips}	& runs full \mips{} test suite	\\
\code{~~~~t\_mips}	& runs tests for \mips{} NLP solver	\\
\code{~~~~t\_mips\_pardiso}	& runs tests for \mips{} NLP solver, using PARDISO as linear solver\tnote{\dag}	\\
\code{~~~~t\_mplinsolve}	& tests for \code{mplinsolve}	\\
\code{~~~~t\_qps\_mips}	& runs tests for \code{qps\_mips}	\\
\bottomrule
\end{tabular}
\begin{tablenotes}
 \scriptsize
 \item [*] {These functions implement new tags and the detection of the corresponding features for \code{have\_feature} which is part of \mptestlink{}.}
 \item [\dag] {Requires the installation of an optional package. See Appendix~\ref{app:pardiso} for details.}
\end{tablenotes}
\end{threeparttable}
\end{table}



%%------------------------------------------
\clearpage
\section{\pardiso{} -- Parallel Sparse Direct and Multi-Recursive Iterative Linear Solvers}
\label{app:pardiso}

The \pardiso{} package is a thread-safe, high-performance, robust, memory efficient and easy to use software for solving large sparse symmetric and non-symmetric linear systems of equations on shared-memory and distributed-memory multiprocessor systems~\cite{pardiso, pardiso2}. More information is available at \url{https://pardiso-project.org}.

When the \matlab{} interface to \pardiso{} is installed, \pardiso{}'s solvers can be used to replace the built-in \code{\textbackslash{}} operator for solving for the Newton update step in \mips{} by setting the \code{linsolver} option equal to \codeq{PARDISO}. The \code{mplinsolve} function can also be called directly to solve $A x = b$ problems via \pardiso{} or the built-in solver, depending on the arguments supplied. This interface also gives access to the full range of \pardiso{}'s options. For details, see \code{help mplinsolve} and the \pardiso{} User's Manual at \url{https://pardiso-project.org/manual/manual.pdf}.


%%------------------------------------------
\clearpage
\section{Release History}
\label{app:release_history}

The full release history can be found in \code{CHANGES.md} or \href{https://github.com/MATPOWER/mips/blob/master/CHANGES.md}{online} at \url{https://github.com/MATPOWER/mips/blob/master/CHANGES.md}.


\subsection{Version 1.0 -- released Feb 7, 2011}
\label{app:v10}

Documentation found in Appendix A of the \href{https://matpower.org/docs/MATPOWER-manual-4.0.pdf}{\matpower{} 4.0 User's Manual}, available online.\footnote{\url{https://matpower.org/docs/MATPOWER-manual-4.0.pdf}}

\subsubsection*{Changes}
\begin{itemize}
\item Licensed under the GNU General Public License (GPL).
\item Added compatibility with GNU Octave, a free, open-source \matlab{} clone.
\item \mips{} (\mipsname{}), a new a pure-\matlab{} implementation of the primal-dual interior point methods from the optional package TSPOPF.
\end{itemize}


\subsection{Version 1.0.1 -- released Apr 30, 2012}
\label{app:v101}

Documentation found in Appendix A of the \href{https://matpower.org/docs/MATPOWER-manual-4.0.pdf}{\matpower{} 4.0 User's Manual}, available online.\footnote{\url{https://matpower.org/docs/MATPOWER-manual-4.0.pdf}}

\subsubsection*{Bug Fixed}
\begin{itemize}
\item Fixed fatal bug in \mips{} for unconstrained, scalar problems. \emph{Thanks to Han Na Gwon.}
\end{itemize}


\subsection{Version 1.0.2 -- released Nov 5, 2013}
\label{app:v102}

Documentation found in Appendix A of the \href{https://matpower.org/docs/MATPOWER-manual-4.0.pdf}{\matpower{} 4.0 User's Manual}, available online.\footnote{\url{https://matpower.org/docs/MATPOWER-manual-4.0.pdf}}

\subsubsection*{Bug Fixed}
\begin{itemize}
\item Fixed a bug in \mips{} where a near-singular matrix could produce an extremely large Newton step, resulting in incorrectly satisfying the relative feasibility criterion for successful termination.
\end{itemize}


\subsection{Version 1.1 -- released Dec 17, 2014}
\label{app:v11}

Documentation found in Appendix A of the \href{https://matpower.org/docs/MATPOWER-manual-5.0.pdf}{\matpower{} 5.0 User's Manual}, available online.\footnote{\url{https://matpower.org/docs/MATPOWER-manual-5.0.pdf}}

\subsubsection*{New Features}
\begin{itemize}
\item Many new user-settable options.
\end{itemize}

\subsubsection*{Incompatible Changes}
\begin{itemize}
\item The name of the \code{mips()} option used to specify the maximum number of step-size reductions with \code{step\_control} on was changed from \code{max\_red} to \code{sc.red\_it} for consistency with other \matpower{} options.
\end{itemize}


\subsection{Version 1.2 -- released Mar 20, 2015}
\label{app:v12}

Documentation found in Appendix A of the \href{https://matpower.org/docs/MATPOWER-manual-5.1.pdf}{\matpower{} 5.1 User's Manual}, available online.\footnote{\url{https://matpower.org/docs/MATPOWER-manual-5.1.pdf}}

\subsubsection*{New License}
\begin{itemize}
\item Switched to the more permissive 3-clause BSD license from the previously used GNU General Public License (GPL) v3.0.
\end{itemize}

\subsubsection*{New Documentation}
\begin{itemize}
\item Added an online function reference to the \matpowerlink{} website at \url{http://www.pserc.cornell.edu/matpower/docs/ref/}.
\end{itemize}

\subsubsection*{New Features}
\begin{itemize}
\item Added support for using \pardiso{} (\url{https://pardiso-project.org/}) as linear solver for computing interior-point update steps in \mips{}, resulting in dramatic improvements in computation time and memory use for very large-scale problems.
\item New functions:
    \begin{itemize}
    \item \code{mplinsolve()} provides unified interface for linear system solvers, including \pardiso{} and built-in backslash operator
    \end{itemize}
\end{itemize}


\subsection{Version 1.2.1 -- released Jun 1, 2016}
\label{app:v121}

Documentation found in Appendix A of the \href{https://matpower.org/docs/MATPOWER-manual-6.0b1.pdf}{\matpower{} 6.0b1 User's Manual}, available online.\footnote{\url{https://matpower.org/docs/MATPOWER-manual-6.0b1.pdf}}


\subsubsection*{Bug Fixed}
\begin{itemize}
\item Fixed issue where default value of \codeq{feastol} option was not being set correctly in \code{mips()} when called directly (or via \code{qps\_mips()}) with \codeq{feastol}\code{ = 0}.
\end{itemize}


\subsection{Version 1.2.2 -- released Dec 16, 2016}
\label{app:v122}

Documentation found in the \href{https://matpower.org/docs/MIPS-manual-1.2.2.pdf}{\mips{} 1.2.2 User's Manual}\footnote{\url{https://matpower.org/docs/MIPS-manual-1.2.2.pdf}} or in Appendix A of the \href{https://matpower.org/docs/MATPOWER-manual-6.0.pdf}{\matpower{} 6.0 User's Manual}, available online.\footnote{\url{https://matpower.org/docs/MATPOWER-manual-6.0.pdf}}

\subsubsection*{New Open Development Model}
\begin{itemize}
% \item \hl{Fill in details from announcement or ``What's New'' section of index.html or README.}
\item \mips{} development has moved to GitHub! The code repository is now publicly available to clone and submit pull requests.\footnote{\url{https://github.com/MATPOWER/mips}}
\item Public issue tracker for reporting bugs, submitting patches, etc.\footnote{\url{https://github.com/MATPOWER/mips/issues}}
\end{itemize}

\subsubsection*{Other Changes}
\begin{itemize}
\item Renamed from \matlab{} Interior Point Solver to \mipsname{}.
\item Remove dependence of \code{t\_mpsolve()} on presence of \code{have\_fcn()} (from \matpower{}) to detect PARDISO installation.
\end{itemize}


\subsection{Version 1.3 -- released Oct 30, 2018}
\label{app:v13}

The \href{https://matpower.org/docs/MIPS-manual-1.3.pdf}{\mips{} 1.3 User's Manual} is available online.\footnote{\url{https://matpower.org/docs/MIPS-manual-1.3.pdf}}

\subsubsection*{New Features}
\begin{itemize}
\item Support for PARDISO 6.x.
\item New \code{mplinsolve} solver option \codeq{LU} for explicit LU decomposition with back substitution, with options in \code{opt.lu} for specifying the number of output arguments in call to \code{lu} (\code{opt.lu.nout}), whether to use permutation vectors or matrices (\code{opt.lu.vec}) and pivot threshold options (\code{opt.lu.thresh}). The following values for the \code{solver} argument act as shortcuts for specifying various combinations of options: \codeq{LU3}, \codeq{LU3a}, \codeq{LU4}, \codeq{LU5}, \codeq{LU3m}, \codeq{LU3am}, \codeq{LU4m}, \codeq{LU5m}.
    See \code{help mplinsolve} for details.
    \emph{Thanks to Jose Luis Mar\'in.}
\end{itemize}

\subsubsection*{Bugs Fixed}
\begin{itemize}
\item Fix bug preventing \code{pardiso.dparm} options from being set.
\end{itemize}

\subsubsection*{Other Changes}
\begin{itemize}
\item \LaTeX{} source code for \mipsman{} included in \code{docs/src}.
\item Move \code{mplinsolve} PARDISO options to \code{opt.pardiso} in preparation for adding options for other solvers.
\end{itemize}


\clearpage
\subsection{Version 1.3.1 -- released Jun 20, 2019}
\label{app:v131}

The \href{https://matpower.org/docs/MIPS-manual-1.3.1.pdf}{\mips{} 1.3.1 User's Manual} is available online.\footnote{\url{https://matpower.org/docs/MIPS-manual-1.3.1.pdf}}

\subsubsection*{Changes}
\begin{itemize}
\item Add \code{CITATION} file.
\item Other miscellaneous documentation updates, e.g. \matpower{} website links updated to \url{https://matpower.org}.
\end{itemize}


\subsection{Version 1.4 -- released Oct 8, 2020}
\label{app:v14}

The \href{https://matpower.org/docs/MIPS-manual-1.3.2.pdf}{\mips{} 1.4 User's Manual} is available online.\footnote{\url{https://matpower.org/docs/MIPS-manual-1.4.pdf}}

\subsubsection*{New Features}
\begin{itemize}
\item Support for \code{have\_feature()} from \mptestlink{} to detect availability and version information for optional functionality. This is a modular, extensible replacement for \code{have\_fcn()} from \matpowerlink{} and \mpomlink{}.
\item Feature detection functions for \code{lu()} and \pardiso{}, defining tags \codeq{lu\_vec}, \codeq{pardiso\_legacy}, \codeq{pardiso\_object} and \codeq{pardiso} for \code{have\_feature()}.
\item New functions:
    \begin{itemize}
    \item \code{have\_feature\_lu\_vec} detects support for the \code{lu(..., \textquotesingle{}vector\textquotesingle{})} syntax.
    \item \code{have\_feature\_pardiso\_legacy} detects support for the legacy (v5.x) \pardiso{} interface, with individual MEX files for factor, solve, etc.
    \item \code{have\_feature\_pardiso\_object} detects support for the object-oriented (v6.x and later) \pardiso{} interface.
    \item \code{have\_feature\_pardiso} detects availability/version of \pardiso{}.
    \end{itemize}
\end{itemize}

\subsubsection*{Bugs Fixed}
\begin{itemize}
\item Silence inadvertent output from \code{mplinsolve()} when called without \code{solver} input argument.
\item Fix fatal errors when \code{mplinsolve()} is called with \codeq{LU} solver and dense \code{A} matrix.
\end{itemize}

\subsubsection*{Other Changes}
\begin{itemize}
\item Requires \mptest{} 7.1 or later.
\item Remove \code{have\_fcn()} dependencies in \code{mips()}, \code{t\_mips\_pardiso()} and \code{t\_qps\_mips()}.
\end{itemize}

\subsubsection*{Incompatible Changes}
\begin{itemize}
\item Calling \code{mips()} with \code{opt.linsolver} set to \codeq{PARDISO} now results in a fatal error if \pardiso{} is not installed, rather than warning and continuing with the default linear solver.
\end{itemize}


\end{appendices}


%%------------------------------------------
\clearpage
%\addcontentsline{toc}{section}{References}
\begin{thebibliography}{99}
\bibitem{tspopf}
TSPOPF. [Online]. Available: \url{http://www.pserc.cornell.edu/tspopf/}.

\bibitem{wang2007a}
H.~Wang, C.~E. Murillo-S{\'a}nchez, R.~D. Zimmerman, and R.~J. Thomas, ``On
Computational Issues of Market-Based Optimal Power Flow,'' \emph{Power
Systems, IEEE Transactions on}, vol.~22, no.~3, pp. 1185--1193, August 2007.
\doi{10.1109/TPWRS.2007.901301}

\bibitem{wang2007}
H.~Wang, \emph{On the Computation and Application of Multi-period Security-constrained Optimal Power Flow for Real-time Electricity Market Operations}, Ph.D. thesis, Electrical and Computer Engineering, Cornell University, May 2007.

\bibitem{zimmerman2011}
R.~D. Zimmerman, C.~E. Murillo-S{\'a}nchez, and R.~J. Thomas, ``\matpower{}: Steady-State Operations, Planning and Analysis Tools for Power Systems Research and Education,'' \emph{Power Systems, IEEE Transactions on}, vol.~26, no.~1, pp.~12--19, Feb.~2011.
\doi{10.1109/TPWRS.2010.2051168}

\bibitem{matpower}
R.~D. Zimmerman, C.~E. Murillo-S{\'a}nchez (2020). \matpower{}\\~
[Software]. Available: \url{https://matpower.org}\\
\doi{10.5281/zenodo.3236535}

\bibitem{mips_manual}
R.~D. Zimmerman, H.~Wang. \mipsname{} (\mips{}) User's Manual. 2020.
[Online]. Available: \url{https://matpower.org/docs/MIPS-manual.pdf}\\
\doi{10.5281/zenodo.3236506}

\bibitem{octave}
John~W.~Eaton, David~Bateman, S{\o}ren~Hauberg, Rik~Wehbring (2015). \emph{GNU Octave version 4.0.0 manual: a high-level interactive language for numerical computations.} Available: \url{https://www.gnu.org/software/octave/doc/interpreter/}.

\bibitem{bsd}
The BSD 3-Clause License. [Online]. Available: \url{https://opensource.org/licenses/BSD-3-Clause}.

\bibitem{gpl}
GNU General Public License. [Online]. Available: \url{https://www.gnu.org/licenses/}.

\bibitem{pardiso}
O.~Shenk and K.~G\"artner, ``Solving unsymmetric sparse systems of linear equations with PARDISO,'' \emph{Journal of Future Generation Computer Systems}, 20(3):475--487,~2004.

\bibitem{pardiso2}
A.~Kuzmin, M.~Luisier and O.~Shenk, ``Fast methods for computing selected elements of the Greens function in massively parallel nanoelectronic device simulations,'' in F.~Wolf, B.~Mohr and D.~Mey, editors, \emph{Euro-Par 2013 Parallel Processing}, Vol.~8097, \emph{Lecture Notes in Computer Science}, pp.~533--544, Springer Berlin Heidelberg, 2013.

\end{thebibliography}


\end{document}
